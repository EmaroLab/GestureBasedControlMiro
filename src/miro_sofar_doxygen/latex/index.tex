\hypertarget{index_aim_sec}{}\section{The aim of the Project}\label{index_aim_sec}
The aim of the project is to control the bio-\/inspired social Robot Mi\+Ro with a wearable device. In particular a smartwatch with a 9-\/axis Imu sensor. ~\newline
 The Robot should also be able to avoid collision with obstacle and override the user\textquotesingle{}s control when an obstacle is detected. ~\newline
 This Project has been developed for the Software Architecture course of the master degree program in Robotics Engineering at University of Genoa. \hypertarget{index_sofar_sec}{}\section{The Software architecture}\label{index_sofar_sec}
The software architecture developed for this project is based on a bio-\/inspired approach. ~\newline
 The problem has been addressed by using a behaviour-\/based design pattern. ~\newline
 ~\newline
 This architecture has been developed to be modular and scalable, since each behavior can be easily modified or substituted, and new behaviours can be easily added. ~\newline
 There are two possible Mi\+Ro behaviors addressed in this project\+: 
\begin{DoxyItemize}
\item Gesture Based behavior 
\item Obstacle Avoidance behavior
\end{DoxyItemize}\hypertarget{index_gbb_sec}{}\subsection{Gesture Based Behavior}\label{index_gbb_sec}
The {\bfseries{Gesture Based behavior}} consists in Mi\+Ro following the user\textquotesingle{}s command. ~\newline
 Hence, the data from the smartwatch\textquotesingle{}s accelerometer are converted into input for Mi\+Ro control. ~\newline
 In particular, depending on the accelerometer values are set some specific lights pattern of Miro\textquotesingle{}s body and Miro\textquotesingle{}s body linear and angular velocities. ~\newline
 Two modalities of control are available\+: {\itshape B\+A\+S\+IC} and {\itshape A\+D\+V\+A\+N\+C\+ED} 
\begin{DoxyItemize}
\item The {\itshape B\+A\+S\+IC} mode is basically a step control that allows Miro to stay still, rotate left/right, go straight forward/backward only. ~\newline
 This mode allows a novice user to start getting comfortable with the control gestures and with the Robot responsiveness. 
\item The {\itshape A\+D\+V\+A\+N\+C\+ED} mode allows a smoother but more sensitive control, enabling combination of basic commands. e.\+g turn left and go forward ~\newline
 This mode allows an expert user to perform more complex trajectories and to exert a more natural control.
\end{DoxyItemize}\hypertarget{index_aob_sec}{}\subsection{Obstacle Avoidance Behavior}\label{index_aob_sec}
The {\bfseries{Obstacle Avoidance behavior}} overrides the Gesture Based behavior when and obstacle is detected by using the Robot\textquotesingle{}s Sonar. ~\newline
 When an Obstacle is detected the Robot\textquotesingle{}s body becomes red to signal the dangerous situation to the user. It starts turning of few degrees until the obstacle is not detected anymore. ~\newline
 But, before this the user must suggest the robot in wich direction turning. Once the collision has been avoided the control goes back to the user. \hypertarget{index_det_sec}{}\subsection{More in details about the architecture and its implementation}\label{index_det_sec}
Each Block of the architecture has been implemente as a R\+OS node. ~\newline
 For comunication between the nodes has been used a Publish/\+Subscibe messaging pattern. ~\newline
 The behavior have been implemented through a miro\+\_\+msg of type \href{https://consequential.bitbucket.io/platform_control.msg}\texttt{ platform\+\_\+control} ~\newline
 In particular, each behavior sets the body\+\_\+vel (Miro Body\textquotesingle{}s velocity) and lights\+\_\+raw (Miro Body\textquotesingle{}s lightening pattern) 